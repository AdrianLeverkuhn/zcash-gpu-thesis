% Chapter Template

\chapter{Related Work} % Main chapter title

\label{Chapter2} % Change X to a consecutive number; for referencing this chapter elsewhere, use \ref{ChapterX}

%----------------------------------------------------------------------------------------
%	SECTION 1
%----------------------------------------------------------------------------------------
\begin{comment}
\section{Background}

Cryptocurrencies have gained in popularity over the last couple of years. The most popular cryptocurrency is Bitcoin. However, Bitcoin is of limited use today. It cannot be used as a general currency due to the low throughput and long waiting times for the transaction to be added to the block. Its status as an anonymous currency is also disputed \cite{biryukov2014deanonymisation, de2017analysis}, resulting in 2 different cryptocurrencies being developed to fill in the void -- Monero\cite{monero} and Zcash \cite{zcashprotocol}.\\
\\
Zcash defines two types of addresses. Transparent addresses behave the same way as Bitcoin addresses. Data resides in the public blockchain so it can be tracked the same way as with Bitcoin. However, shielded addresses reveal nothing when they are a part of the transaction. This is accomplished by encrypting the transactions and providing a zero-knowledge proof of correctness of the transaction \cite{zcashtechnology}.\\
\\
However, the zero-knowledge proof used (zk-SNARK \cite{ben2014succinct, groth2016size}) requires a substantial amount of time to compute, limiting the adoption of shielded transactions on portable devices. The most computationally intensive part of zk-SNARKs is multiexponentiation. It involves multiplication of thousands of elliptic curve points by a scalar and the addition of the results. Due to its inherently parallel nature, it may be possible to gain a speed boost by using a GPU built in modern PCs and phones.
\end{comment}


%\section{Related Work}

Wu et al \cite{wu2018dizk} implemented a distributed version of zk-SNARKs. Their implementation increased the supported circuit size from 10-20 million to 1 billion. This was done by distributing work on computing clusters. Their result shows that zk-SNARK computation is quite parallelizable. However, Wu et al focused on distributing proof calculation on multiple CPUs and clusters, not GPUs.\\
\\
Elliptic curve operations have been frequently benchmarked on GPUs \cite{mahefast, bernstein2010ecc2k, antao2010elliptic}. However, all of the papers focused on throughput of scalar multiplication (exponentiation) used in cryptographic signatures. The consensus is that GPUs can lead to a significant speedup when calculating many signatures. While exponentiation algorithms can be used to implement multiexponentiation, multiexponentiation is a problem of great importance in cryptography, and should be treated separately.\\
\\
Chang and Lou \cite{chang1997parallel} and Borges et al \cite{borges2017parallel} took a look at multiexponentiation in a parallel setting. Chang and Lou showed that multiexponentiation can be efficiently distributed to multiple nodes. Borges et al benchmarked parallel multiexponentiation on multi-core processors in a modular group and reported a significant speedup. However, none of these papers discuss accelerating elliptic curve multiexponentiation on a GPU.\\
\\
Many papers compare OpenCL performance to CUDA, or accross two different platforms -- usually NVIDIA and AMD \cite{fang2011comprehensive, karimi2010performance}. They usually focus on kernel performance and tuning the kernels to run efficiently on different platforms \cite{komatsu2010evaluating, henry2014toward}. This is in line with Sorensen's  paper \cite{sorensen2016hitchhiker} where they show that that developers rarely test their code on multiple platforms, and that platforms often suffer from compatibility issues and bugs. In this thesis we focus on issues programers need to overcome to develop a kernel for devices from four different vendors (NVIDIA, AMD, Intel, ARM).