% Chapter 1

\chapter{Introduction} % Main chapter title

\label{Chapter1} % For referencing the chapter elsewhere, use \ref{Chapter1} 

%----------------------------------------------------------------------------------------

% Define some commands to keep the formatting separated from the content 
\newcommand{\keyword}[1]{\textbf{#1}}
\newcommand{\tabhead}[1]{\textbf{#1}}
\newcommand{\code}[1]{\texttt{#1}}
\newcommand{\file}[1]{\texttt{\bfseries#1}}
\newcommand{\option}[1]{\texttt{\itshape#1}}

%----------------------------------------------------------------------------------------

One of the biggest events in computer science in the last decade was the invention of Bitcoin \cite{nakamoto2008bitcoin}. On the surface, Bitcoin offers perfect anonymity. Users can generate an arbitrary number of new addresses. Many parties also offer tumblers that transfer Bitcoin through thousands of different accounts and send laundered funds to the user (for a small fee). However, data in the blockchain is public. Transaction history can be combined with out-of-blockchain data to de-anonymize users of Bitcoin \cite{biryukov2014deanonymisation}. Further graph analysis can be used to defeat tumblers as well \cite{de2017analysis}.\\
\\
ZCash \cite{zcashprotocol} is a fork of Bitcoin that tries to address this issue. It contains two types of addresses -- transparent (\textbf{t-addr}) and shielded (\textbf{z-addr}). Transparent addresses behave like Bitcoin addresses -- all transaction history (identities and amounts) are public. Shielded addresses encrypt this data to prevent leaks -- the transactions reveal nothing about its users, or the amounts transferred. For transparent addresses, miners can easily check if the transaction is valid (eg. the account has enough money) by iterating through the previous transactions in the blockchain. For shielded addresses this isn't possible, so the party creating the transaction needs to provide one more piece of information -- a zero-knowledge proof that the transaction is valid. \\
\\
It isn't enough for a proving system to be zero-knowledge to be used in practice. Its proofs need to be small because it will be stored in the blockchain. Furthermore, miners need to verify every transaction before they add it to the block, so it must be non-interactive and fast to verify. ZCash uses zk-SNARKS for this purpose, but these properties come at a cost -- proof generation is extremely slow. That's why many wallets don't support shielded transactions. Considering that many users have Bitcoin wallets on their phones, which are slow compared to desktop CPUs, this is preventing more widespread use of zCash.\\
\\
In this thesis we take a look at using graphics cards, present on many devices today, to accelerate zk-SNARKs. In order to make our solution cover as many platforms as possible (including mobile phones), we port performance critical code (scalar multiexponentiation over curve BLS12-381) to OpenCL \cite{stone2010opencl}. We compare the differences, as well as difficulties in running cross-platform OpenCL code. We benchmark different algorithms for multi-exponentiation on different devices (Intel, NVIDIA , AMD and ARM), and compare the results.\\
\\
The remainder of the thesis is organized as follows. The background and related research are presented in Chapter \ref{Chapter2}. In Chapter \ref{Chapter3} we explain the anatomy of zCash and zk-SNARKs. Chapter \ref{Chapter4} covers the architecture of OpenCL. Implementation details of different algorithms are covered in Chapter \ref{Chapter5}. Chapter \ref{Chapter6} describes the tests we performed and the results of our measurements, as well as the comparison of OpenCL support by different vendors. Chapter \ref{Chapter7} discusses the results from Chapter \ref{Chapter6}. Finally, we propose possibilities for further research in Chapter \ref{Chapter8} and provide a summary of our most important results in Chapter \ref{Chapter9}.
%----------------------------------------------------------------------------------------
